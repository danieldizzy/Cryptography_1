\documentclass{article}
\title{Problem 3: The Sage Zone}
\author{UW Student who knows SageTex!}
\date{2016-04-22}
\usepackage{sagetex}

\begin{document}
\maketitle

\section{Factoring Years\label{years}}
Sage says\footnote{These factorization are
computed using sagetex!} that $2016=\sage{factor(2016)}$ and
$2017=\sage{factor(2017)}$.

\section{Plotting a Function\label{plotting}}
Here is a plot of $\sin(x^2)$ made using sagetex.  Your
plot should be about this size (not enormous).
\vspace{.5in}

\sageplot[width=.7\textwidth]{plot(sin(x^2), 0, pi)}

\section{Deriving a Formula\label{formula}}

Sage can find a formula for
$
 f(n) = \sin(1) + \sin(2) + \cdots + \sin(n).
$
Just enter this code into Sage (in sagetex use the sageblock environment):
\begin{sageblock}
var('k, n')
f = sum(sin(k), k, 1, n)
\end{sageblock}
and find that
{\tiny
$$
f = \sage{f}
$$
}

Here is a plot of the formula above from $0$ to $100$:

\sageplot[width=\textwidth]{plot(f, 0, 100, color='red')}



\end{document}
